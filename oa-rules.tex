\documentclass[a4paper,titlepage,12pt]{article}
\usepackage[utf8x]{inputenc}
\usepackage[russian]{babel}
\usepackage{geometry}
\usepackage{amsmath}

\geometry{top=1cm}
\geometry{bottom=1cm}
\geometry{left=0.5cm}
\geometry{right=1cm}

\begin{document}

{\center\section*{Как определять результат игры OpenArena}}

\thispagestyle{empty}

\begin{enumerate}
\item Игра продолжается от начала, когда счет игроков обнулен,
    до завершения.
    Игра состоит из серии схваток.
\item Если в этих правилах будут найдены противоречия или неоднозначности,
    результат всех игр в тот день~--- отмена.
\item Оговорить что-либо (=договориться о чем-либо)~---
    значит получить от всех участников игры
    письменное подтверждение этого с датой и подписью
    или аудиозапись с устным подтверждением этого,
    в которой была бы названа текущая дата.
    В противном случае договоренности не было,
    кроме тех случаев, когда оба игрока считают, что она была.
\item Данные правила вступают в силу в момент, когда оба игрока
    их подписали, и перестают учитываться, когда новая версия этих правил
    (с более поздней датой) будет подписана обоими игроками.
\item Результат игры определяют по версии правил, действительной в момент
    начала игры.
\item Данные правила применимы только для игры двух людей
    (с ботами или без них) в игру OpenArena.
    В остальных случаях, если не было оговорено особых правил,
    результат этой игры~--- отмена.
\item Допускается оговаривать особые правила, если они не противоречат
    данным правилам.
\item Перед каждой игрой желательно оговаривать, идет ли игра на счет.
    Если такой договоренности перед игрой не было,
    результат этой игры~--- отмена.
\item Ничья или отмена игры означает, что никто не победил и никто не проиграл.
    Игрок, заявляющий, что он играл лучше в отмененной или ничейной игре,
    нарушает правила.
\item Нельзя утверждать, что какой-то игрок играет лучше, чем другой игрок
    без согласования с обоими этими игроками.
\item Нарушение данных правил засчитыванием ему одной проигранной игры
    (не более 1 раза в сутки).
\item Если игра идет на счет, во время игры нельзя менять условие победы,
    состав игроков и условие завершения игры.
\item Если игра идет без команд и играют на счет, для определения
    победителя в игре в качестве оценки используется счет игрока,
    если не было оговорено другой оценки.
\item Если игра идет с командами и играют на счет, для определения
    победителя в игре в качестве оценки используется корень из
    произведения счета своей команды и суммы своего счета и счетов
    всех ботов врага, если не было оговорено другой оценки.

    $$ \sqrt{\text{счет своей команды} \cdot (\text{свой счет} +
        \sum\limits_{\text{бот врага}}\text{счет бота врага})} $$

\item Если играли на счет и оценки игроков составили A и B очков,
    ничейной будет игра, в которой функция
    $ \text{BINOMDIST}(A, B, 0.5, 1) $ (из программ
    OpenOffice.org Calc, Libre Office Calc, Gnumeric или Excel)
    вернет результат в интервале $ (0.05, 0.95) $.
    Если играли на счет и эта функция вернула результат, не принадлежащий
    интервалу $ (0.05, 0.95) $, побеждает игрок, у которого оценка выше.
\item Очки считаются до заранее оговоренного условия завершения игры
    (к примеру, достижения 50 очков одним из игроков или прошествии 30 минут)
    или до ситуации, когда оба игрока не могут продолжать игру.
    Если условие завершения не было оговорено,
    результат этой игры~--- отмена.
    Если оговоренное условие не наступило, но один из игроков не может
    продолжать игру в течении 5 минут или более, он считается проигравшим
    вне зависимости от очков.

\end{enumerate}

Дата: \today

Игровое прозвище и подпись:

Игровое прозвище и подпись:

\end{document}

